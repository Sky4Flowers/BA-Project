\chapter{\abstractname}

%English abstract text.

Due to the developments in the fields of virtual and augmented reality in the recent years, which brought a variety of different realizations of so-called head-mounted displays, it appears these areas could play an increasingly important role in everyday life in the future. And this requires, among other things, intuitive user interfaces. While capturing controllers and the user's hands offers options for user interface placement, previous cross-platform applications for head-mounted displays have not exploited this potential yet.
Therefore, this work deals with the topic of placing user interfaces on body positions, which can be tracked by the program, and more specifically with the handling of loosing such positions. For this purpose, a concept for realizing the redistribution of user interfaces between detected body positions was developed and later evaluated with a user study.

\vspace{0.5in}

\noindent %German abstract text.

Durch die Entwicklungen der letzten Jahre in den Gebieten der virtuellen und erweiterten Realität, welche eine Vielzahl an verschiedenen Umsetzungen von sogenannten Head-Mounted-Displays mit sich brachten, wird klar, dass diese Bereiche in Zukunft eine immer größere Rolle im Alltag spielen könnten. Und dazu werden unter anderem intuitive Benutzeroberflächen benötigt. Zwar bietet die Erfassung von Controllern und den Händen des Nutzers Optionen zum Platzieren von Nutzeroberflächen an, aber in bisherigen plattformübergreifenden Anwendungen für Head-Mounted-Displays wird dieses Potential nicht genutzt.
Aus diesem Grund beschäftigt sich diese Arbeit mit dem Thema der Platzierung von Benutzungsoberflächen an Körperpositionen, welche durch das Programm erfasst werden können, und genauer mit der Behandlung des Wegfallens solcher Positionen. Dafür wurde ein Konzept zur Realisierung der Umverteilung von Benutzeroberflächen zwischen erfassten Körperpositionen entwickelt und daraufhin mittels einer Nutzerstudie bewertet.


\makeatletter
\ifthenelse{\pdf@strcmp{\languagename}{english}=0}
{\renewcommand{\abstractname}{Zusammenfassung}}
{\renewcommand{\abstractname}{Abstract}}
\makeatother

%\chapter{\abstractname}


% Undo the name switch
\makeatletter
\ifthenelse{\pdf@strcmp{\languagename}{english}=0}
{\renewcommand{\abstractname}{Abstract}}
{\renewcommand{\abstractname}{Zusammenfassung}}
\makeatother