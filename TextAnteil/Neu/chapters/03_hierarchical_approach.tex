% !TeX root = ../main.tex

\chapter{Problemdiskussion}\label{chapter:dimensionwise_refinement}

	%References for chapters work like this. blblbla see \refChapter{chapter:background}.
			
	%\section{Program Structure}
	
		%For program code, use the following structure:
		
		%\begin{figure}[htbp]
			%\centering
			%\begin{tabular}{c}
			%\begin{lstlisting}[language=python]
			%	# enter your code here (# marks a comment in python)
			%	def (stuff as stuff):
			%		x = 1
			%	\end{lstlisting}
			%\end{tabular}
			%\caption{Caption Example}
			%\label{fig:code_integrator}
		%\end{figure}
		
	\section{Problembeschreibung}
		Da es viele verschiedene Umsetzungen der virtuellen und erweiterten Realität gibt, steht man als Entwickler eines Mixed-Reality Programms über Kurz oder Lang vor einigen Hindernissen. Eines davon ist die Platzierung der Benutzeroberfläche...
		Bisher wird diese in einem neuen Projekt immer wieder von Neuem, der Situation angepasst, platziert.
		\todo{Bis hier überarbeiten}
		Besondere Schwierigkeiten entstehen aber noch zusätzlich, wenn man das Programm auf verschiedenen Geräten nutzen möchte. So werden zum Beispiel bei vielen Varianten der erweiterten Realität die Handpositionen nie oder zumindest nicht kontinuierlich verfolgt und ermöglichen somit kein konstantes Anzeigen von virtuellen Elementen an diesen Stellen. Sie können daher schlecht als Ankerpositionen für die Benutzeroberfläche verwendet werden.
		Andere Geräte stellen durch die Positionsbestimmung \todo{Finde Formulierung für Controller und VrBrille und an allen Stellen übernehmen} Hand- und Kopfpositionen zu jeder Zeit zur Verfügung. Das Wegfallen von Controllern durch niedrigen Akkustand oder andere Umstände erzeugt dabei allerdings ein ähnliches Problem wie bei den Varianten ohne solche Controller.
		Da die Positionierung der Benutzeroberfläche an den Handpositionen deutlich mehr Möglichkeiten eröffnet \todo{Beweis?}, sollten diese auch genutzt werden, solange sie vorhanden sind. Für den Fall ihrer Abwesenheit ist es dann allerdings nötig eine Rückfallmöglichkeit bereitzustellen, damit keine Elemente der Benutzeroberfläche verloren gehen und mit ihnen deren Funktion und Information...
		
	\section{Ansatz}
		Um die Verwendung von allen Elementen der Benutzeroberfläche auch beim Verlust einer oder mehrerer möglicher Ankerpositionen zu gewährleisten, wurden in dieser Arbeit mehrere Möglichkeiten erarbeitet, um eine Umverteilung der enthaltenen Informationen von einem nicht verwendeten Anker auf einen oder mehrere aktive Anker durchführen zu können.
		Dafür wird betrachtet wie wichtig die einzelnen Inhalte sind, welche Ankerpositionen zur Verfügung stehen und was für eine Rückfallart bei der Verschiebung verwendet werden soll. Letztere entscheidet darüber wie der Rückfall aufgefangen wird.
		Die Neuplatzierung der umverteilten Bausteine übernimmt jeweils der neue Anker.
		\todo{fortsetzen}
		
		\subsection{Umsetzung der Prioritätsstufen}
			Um herauszufinden wie wichtig ein Element ungefähr ist, werden erst klare Klassen benötigt, welche unterschiedliche Prioritäten darstellen.
			- vorbild S.H.I.T.
			- wichtig für Verarbeitung bei Fallback
			- High/Medium/Low/None
			
		\subsection{Erweiterung}
			- Switch
			- Directional
		
		\subsection{Rückfallarten}
			- Über ID
			- Über Container
			- Innerhalb eines Containers
			
		\subsection{TODO}
			\todo{Bereich benennen}
			
	\section{Umsetzung}
		- Erklärung der Struktur
			
		\subsection{UIAnchor}
			- stellt Anker dar
			- gibt Anweisungen an Kinderanker weiter
			- Manipuliert Elemente abhängig von Ankertyp
			- Setzt Fallback um
			
			- Cylinder (Beispiel Bild)
			- Rectangle 
			- Umsetzungen abhängig von Prio und Position
			- Vorbild Rectangle Hololens
			- Cylinder Abwandlung von Planetarium und Ähnlichem
		
		\subsection{UIContainer}
			- enthält Elemente
			- ermöglicht strukturierte Platzierung von dynamischem Inhalt
		
		\subsection{AnchoredUI}
			- Stellt Element dar
			- Beinhaltet Informationen für die Fallback Umsetzung
		
		\subsection{AnchorManager}
			- Regelt das Wegfallen von Ankern
			- Weist FBAnker zu
			
	\section{Evaluierung}
		%- Studie
		%- 13 Teilnehmer
		Um die beschriebene Umsetzung zu evaluieren wurde eine Nutzerstudie durchgeführt, an der dreizehn Freiwillige im Alter zwischen ... und ... teilnahmen. \todo{Altersbereich herausfinden}
	
		\subsection{Studienablauf}
			%- 4 Szenarien
			In der Studie mussten die Probanden in vier verschiedenen Szenarien jeweils eine Reihe von Aufgaben durchführen. Die Reihenfolge der Szenarios wurde dabei jedes Mal variiert und war für jeden Teilnehmer unterschiedlich, um einen Einfluss der Reihenfolge auf das Gesamtergebnis möglichst gering zu halten. Die Aufgabenstellung blieb in allen Szenarien die Gleiche.
			
			
			
			Zu Anfang sollten die Tester ein paar einleitende Fragen beantworten, dann führten sie die Aufgaben ein erstes Mal in einer der \todo{Anderes Wort für Szenarien} und beantworteten danach Fragen zu dem Szenario und einzelnen Elementen darin.
			- Aufzählung der Fragen
			- Wie wurden Elemente ausgewählt?
			Dies wurde viermal wiederholt. Zum Abschluss sollten sie noch angeben, wie schwierig sie die Aufgabenstellung und die Steuerung unabhängig von den Szenarien fanden und in welche der Prioritätskategorien (siehe \todo{Verlinkung zu Prioritäten}) sie die ausgewählten Elemente einordnen würden.
		
		\subsection{Szenario-Beschreibung}
			- Szenenbeschreibung
			- verwendete UI Elemente
			- 
		
		