% !TeX root = ../main.tex
% Add the above to each chapter to make compiling the PDF easier in some editors.

\chapter{Einleitung}\label{chapter:introduction}

	%Calculating the integral of a function $f(x)$ is a typical problem in numerics. In many cases, the function $f(x)$ is not known but only the function values at certain, potentially arbitrary, points. In this scenario, a common approach is to approximate the integral using the finite set of sampling points. This is a non-trivial problem.
	
	%The term \term{term} is marked like this for better understanding.
	
	%\todo{This marks a todo}{}

	\section{Was ist eine Benutzeroberfläche?}
	- Schnittstelle zwischen Nutzer und Programm
	-Stellt Information zur Verfügung und ermöglicht die Erfassung und Manipulation der virtuellen Umgebung
	
		\subsection{Element}
		- Beispiele: Textfeld, Bild, Schaltfläche...
	
		\subsection{Anker}
		- Position an welcher Elemente der Benutzeroberfläche platziert werden
		- In Desktop-Programmen werden dafür meist die Bildschirmkanten beziehungsweise Ecken verwendet
		-Beispiel aus Unity
	
	\section{Was bedeutet umgebungsabhängig?}
	- UI befindet sich im 3D Raum und kann von Objekten verdeckt werden. Sie ist an eine 3D Position gebunden und kann sich unabhängig vom Benutzer verschieben und rotieren, sie muss also nicht statisch zur Kamera sein
	- In 3D Programmen wird solche UI meist an Objekten positioniert und ist zu diesen statisch
	-In dieser Arbeit wurden als Ankerpunkte die beiden Handpositionen des Benutzers und die Kopfposition verwendet
	
	\section{Was ist VR/AR?}
	- Head Mounted Display
	- Mischung von Realität und virtuellen Elementen
	- Augmented Reality = Erweiterte Realität
	- Virtual Reality = Virtuelle oder künstliche Realität
	- AR lebt von der realen Welt und somit von freiem Bewegen
	-> nicht an Pc gebunden
	--> weniger Leistung
	--> kein Handtracking