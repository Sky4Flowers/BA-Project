% !TeX root = ../main.tex

\chapter{Ergebnisse}\label{chapter:results}
	Dieses Kapitel gibt die Ergebnisse der Studie wieder und vergleicht diese mit den ursprünglichen Annahmen. Dazu werden zuerst die angenommenen Prioritäten mit den Bewertung durch die Tester gegenübergestellt.

	\section{Prioritäten}
		Die nachfolgende Tabelle (siehe \autoref{tab:priority}) stellt die Nutzerbewertung der, in \refChapter{chapter:szenario} erwähnten, Elemente der Benutzeroberfläche dar. Die einzelnen Abstufungen werden wie folgt definiert:
		
		\todo{Refchapter prios}
		\begin{table}[htpb]
			\caption[Example table]{Bewertung der Wichtigkeit von Elementen aus den Szenarien} \label{tab:priority}
			\centering
			\begin{tabular}{l c c c c c}
				\toprule
				Elementbezeichnung & unnötig & unwichtig & neutral & wichtig & sehr wichtig \\
				 & (1) & (2) & (3) & (4) & (5)\\
				\midrule
				Inventar & 1 & 0 & 9 & 2 & 1\\
				Ausrüstung & 0 & 0 & 5 & 4 & 4\\
				Cursor & 0 & 0 & 1 & 0 & 12\\
				Ausgewähltes Objekt & 0 & 1 & 2 & 6 & 4\\
				Switch & 1 & 1 & 7 & 3 & 1\\
				Handelsobjekt & 0 & 7 & 2 & 3 & 1\\
				Schließen & 0 & 5 & 2 & 3 & 3\\
				Vermögen & 0 & 3 & 5 & 3 & 2\\
				\bottomrule
			\end{tabular}
		\end{table}
		
		Ein Objekt ist \term{sehr wichtig}, wenn es zu jedem Zeitpunkt angezeigt wird und sich ebenso immer im Sichtfeld des Nutzers befindet. Es kann sich somit weder verbergen lassen, noch sich aus dem Sichtfeld bewegen. Diese Stufe ist äquivalent mit der Priorität \term{high}, welche in Kapitel \todo{Referenz} erläutert wird. Kontextabhängige Varianten davon bilden eine Ausnahme, da sie nur in bestimmten Situationen als \term{sehr wichtig} gelten und deshalb nur dort auf die beschriebene Weise angezeigt werden. In anderen Situationen sind diese Informationen versteckt.
		
		Als \term{wichtig} gilt ein Element allerdings bereits, wenn es lediglich zu jeder Zeit angezeigt wird, aber nicht immer sichtbar ist. Es kann sich also auch aus dem Sichtfeld hinaus bewegen, aber lässt sich auf keine Weise ausschalten. \todo{Vergleichbar mit Medium} Wie bei der Stufe \term{sehr wichtig} bilden auch hierbei kontextabhängige Objekte die selbe Ausnahme.
		
		\term{Neutral} sind ...
		Die Objekte der Priorität \term{low}, welche vom aktuellen Kontext abhängen, werden als eigene Stufe gehandhabt. Sie werden als \term{unwichtig} bezeichnet.
		
		Die letzte Stufe ist äquivalent zu der Priorität \term{none}. Bei diesen, als \term{unnötig} erachteten, Elementen macht es keinen Unterschied für die Bewältigung der Aufgabe, ob diese Elemente vorhanden sind oder nicht.
		
		
		Es ist eindeutig erkennbar...
	
	\section{Bewertung der Szenarien}