% !TeX root = ../main.tex

\chapter{Fazit}\label{chapter:conclusion}

	Im Nachfolgenden werden die wichtigsten Ergebnisse dieser Arbeit kurz zusammengefasst und es wird ein Ausblick darüber gegeben, was man in Zukunft potentiell an dem Konzept verbessern könnte, beziehungsweise inwiefern man es erweitern kann. Dies beinhaltet ebenso die Grenzen und Schwächen des in dieser Arbeit entwickelten Konzepts.

	\section{Zusammenfassung}
	
		Die Nutzerstudie hat gezeigt, dass das Konzept zur Platzierung von Benutzeroberflächen an einer variablen Anzahl von verfügbaren Ankerobjekten, welches in dieser Arbeit entwickelt wurde, noch Verbesserungspotential hat.
		Besonders Benutzeroberflächen, die immer sichtbar sind, wurden negativ bewertet und müssen auf eine andere Weise realisiert werden, um den Nutzer nicht zu behindern.
		Auch Elemente, die anfangs nicht angezeigt werden und erst später erscheinen, brachten die Tester durcheinander. An diese müssen Benutzer also erst herangeführt werden, damit sie wissen, wo welche Information zu finden ist.
		Im Groben bewerteten die Teilnehmer das Konzept aber positiv.
		Besonders das Objekt in der Mitte des Sichtfelds, das anzeigt, ob der Benutzer mit etwas interagieren kann, stach dabei heraus. Und das obwohl die Tester die gewählte Art der Interaktion nicht intuitiv fanden.
		Für weitergehende Studien sollte deshalb eine andere Weise der Interaktion gefunden werden.
		%- Steuerung war für einige Nutzer nicht intuitiv aber mit Übung ging es
		%- Umsetzung fanden die Tester im Groben gut?
		%- mitbewegende DauerUI wurde negativ bewertet, zu groß, zu nah
		%- Schwierigkeiten mit verborgenen Elementen
		Zum Abschluss muss aber auch noch erwähnt werden, dass die Studie nur 13 Teilnehmer umfasst und aus diesem Grund nur wenig Aussagekraft hat.
		%- Relativ wenige Tester
		
	
	\section{Ausblick}
		
		In der Zukunft wäre eine umfangreichere Studie nützlich, um die Wirkung einzelner Elemente auf die Nutzer besser zu erfassen und ein genaueres Abbild der durchschnittlichen Nutzermeinung zu erhalten. Dazu ist es nötig weitere Altersgruppen miteinzubeziehen, wie zum Beispiel mehr Tester, welche zwischen 13 und 18 oder 25 und 40 Jahren alt sind. Die bisherige Studie umfasst nämlich lediglich den Altersbereich 16 bis 26. Außerdem muss die Anzahl der Teilnehmer erhöht werden.
		
		
		Des weiteren kann die Umsetzung einiger Elemente des, über diese Arbeit erstellten, Programms verbessert werden. So wäre es zum Beispiel sinnvoll, die einzelnen zweidimensionalen Objekte eines Zylinderankers so zu verformen, dass sie tatsächlich auf der Oberfläche des Zylinders angezeigt werden und somit selbst unterschiedlich große Elemente nahtlos nebeneinander platziert werden können. Bisher sind sie noch planar und nur ihr Mittelpunkt  liegt auf dem Zylinder, was besonders bei großen Elementen dazu führt, dass deren seitliche Kanten deutlich weiter vom Zylindermittelpunkt und somit vom Standpunkt des Nutzers entfernt sind.
		Denkbar wäre es außerdem das Prinzip des Zylinders ebenso auf eine Kugel anzuwenden.
		
		
		Zudem fehlen bisher noch alternative Implementierungen von Containern neben den Ankern, welche eine sinnvolle und einfache Umlagerung von Bausteinen zwischen zwei verschiedenen Ankern ermöglichen. Möglich wären auch Container für andere Container.
				
		
		Ein weiteres wichtiges Thema, welches in dieser Arbeit nicht behandelt wurde, aber in diesem Bereich in Zukunft eine große Rolle spielen wird, ist die Begrenzung des Informationsumfangs an einem einzelnen Anker. Dies ist nämlich ein wichtiger Punkt im Hinblick auf das Nutzererlebnis. Zu viel Information kann den Nutzer überfordern beziehungsweise stören, wie man bei der Auswertung der Umsetzung mit nur der linken Hand in \refChapter{chapter:resultsSzen} sieht.
		
		
		Zur Erleichterung der Interaktion wäre die Entwicklung verschiedener Interaktionsformen für die einzelnen Szenarien praktisch. Die in dieser Arbeit verwendete Form war nämlich nicht sonderlich intuitiv und nutzt das Potential der Controller nicht.
		
		
		Die Probleme des umschaltbaren Menüs könnte man durch eine klare Beschriftung eventuell verbessern. Das Anbringen eines kleinen Hinweises, welcher am Rand des Sichtfelds auftaucht, könnte das Finden verborgener Elemente ebenfalls weiter vereinfachen.
	
		
		%-Verformen der UI
		%-Containerimplementierung
		%-Wie viel UI?