 !TeX root = ../main.tex

\chapter{Fazit}\label{chapter:conclusion}

	Im Nachfolgenden werden die wichtigsten Ergebnisse dieser Arbeit kurz zusammengefasst und es wird ein Ausblick darüber gegeben, was man in Zukunft potentiell an dem Konzept verbessern könnte, beziehungsweise inwiefern man es erweitern kann. Dies beinhaltet ebenso die Grenzen und Schwächen dieser Arbeit.

	\section{Zusammenfassung}
		- Relativ wenige Tester
		- Steuerung war für einige Nutzer nicht intuitiv
		- Umsetzung fanden die Tester im Groben gut?
		- Schwierigkeiten mit verborgenen Elementen
	
	\section{Ausblick}
		
		In der Zukunft wäre eine umfangreichere Studie nützlich, um die Wirkung einzelner Elemente auf die Nutzer besser zu erfassen und ein genaueres Abbild der durchschnittlichen Nutzermeinung zu erhalten. Dazu ist es nötig weitere Altersgruppen miteinzubeziehen, wie zum Beispiel Tester, welche zwischen 13 und 18 oder 25 und 40 Jahren alt sind. Die bisherige Studie umfasst nämlich lediglich den Altersbereich \todo{Altersbereich}. Außerdem muss die Anzahl der Teilnehmer erhöht werden.
		
		
		Des weiteren kann die Umsetzung einiger Elemente des, über diese Arbeit erstellten, Programms verbessert werden. So wäre es zum Beispiel sinnvoll, die einzelnen zweidimensionalen Elemente \todo{Zu oft Elemente!} eines Zylinderankers so zu verformen, dass sie tatsächlich auf der Oberfläche des Zylinders angezeigt werden und somit selbst unterschiedlich große Elemente nahtlos nebeneinander platziert werden können. Bisher sind sie noch planar und nur ihr Mittelpunkt  liegt auf dem Zylinder, was besonders bei großen Elementen dazu führt, dass deren seitliche Kanten deutlich weiter vom Zylindermittelpunkt und somit vom Standpunkt des Nutzers entfernt sind.\todo{Graphisch darstellen}
		\todo{Mehr!}
		- Sphere Mapping
		
		
		Außerdem fehlen bisher noch alternative Implementierungen von Containern neben den Ankern, welche eine sinnvolle und einfache Umlagerung von Bausteinen zwischen zwei verschiedenen Ankern ermöglichen. Denkbar wären auch Container für Container.
				
		
		Ein weiteres wichtiges Thema, welches in dieser Arbeit nicht behandelt wurde, aber in Zukunft sicherlich eine große Rolle spielen wird, ist die Begrenzung des Informationsumfangs an einem einzelnen Anker. Dies ist nämlich ein wichtiger Punkt im Hinblick auf das Nutzererlebnis. \todo{Verweis!} Zu viel Information kann den Nutzer überfordern beziehungsweise stören.
		
		
		- Verschiedene Interaktionsmethoden für die Szenarien entwickeln
		
		- Verdecken besser umsetzen
		
		%-Verformen der UI
		%-Containerimplementierung
		%-Wie viel UI?