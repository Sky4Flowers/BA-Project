% !TeX root = ../main.tex

\chapter{Fazit}\label{chapter:conclusion}

	Im Nachfolgenden werden die wichtigsten Ergebnisse dieser Arbeit kurz zusammengefasst und es wird ein Ausblick darüber gegeben, was man in Zukunft an dem Konzept verbessern könnte, beziehungsweise inwiefern man es erweitern kann.
	%This chapter summarizes the important lessons that can be learned from this and gives an outlook on how to potentially improve the concept.

	\section{Zusammenfassung}
		- Relativ wenige Tester
		- Steuerung war für einige Nutzer nicht intuitiv
		- Umsetzung fanden die Tester im Groben gut?
		- Schwierigkeiten mit verborgenen Elementen
	
	\section{Ausblick}
		
		In der Zukunft wäre eine umfangreichere Studie nützlich, um die Wirkung einzelner Elemente auf die Nutzer besser zu erfassen und ein genaueres Abbild der durchschnittlichen Nutzermeinung zu erhalten. Dazu ist es nötig weitere Altersgruppen miteinzubeziehen. - Mehr Altersgruppen 25+ und 15-
		
		Des weiteren kann die Umsetzung einiger Elemente meines Programms verbessert werden. So wäre es zum Beispiel sinnvoll, die einzelnen zweidimensionalen Elemente \todo{Zu oft Elemente!} eines Zylinderankers so zu verformen, dass sie tatsächlich auf der Oberfläche des Zylinders angezeigt werden und somit selbst unterschiedlich große Elemente nahtlos nebeneinander platziert werden können. Bisher liegt nur ihr Mittelpunkt auf dem Zylinder, was besonders bei großen Elementen dazu führt, dass deren seitliche Kanten deutlich weiter vom Zylindermittelpunkt und somit vom Standpunkt des Nutzers entfernt sind.
		\todo{Mehr!}
		
		Außerdem fehlen bisher noch alternative Implementierungen von Containern neben den Ankern, welche eine sinnvolle und einfache Umlagerung von Bausteinen ... zu einem anderen Anker ermöglichen. Denkbar wären auch Container für Container.
				
		-Verformen der UI
		-Containerimplementierung
		-Wie viel UI?