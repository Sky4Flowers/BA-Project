\chapter*{List of Notations} \addcontentsline{toc}{chapter}{List of Notations}  

\begin{center}
	\begin{tabularx}{\linewidth}{lX}
		\toprule
		Notation						&	Description\\
		\midrule
		$\R$							&	The real numbers including the zero.\\
		%$\N$							&	The natural numbers.\\
		$\vec{u}$						&	The vector $u$. \\
		$u_i$							&	$i$th element of vector $\vec{u}$.\\
		$\bigO$							&	Big O Notation.\\
		$|x|_n$							&	The $l_n$ norm of $x$. \\
		%\bottomrule
%	\end{tabularx}
%
%	Specific term notations were used as following table shows:
%	\begin{tabularx}{\linewidth}{lX}
%		\toprule
%		Notation						&	Description\\
		\midrule
		$\Phi(x)$						&	The \term{hat function}. \\
		$x_i$							&	Sampling point with index $i$.\\
		$\Phi_i(x)$						&	The \term{hat function} for the point with index $i$. \\
		$h$								&	The mesh size. \\
		$d$								&	Number of dimensions. \\
		$\ell$							&	The discretization level.\\
		$w$								&	The width of a basis function.\\
		$s_i$							&	The support of the \term{hat function} from point with index $i$.\\
		$\alpha_{i}$					&	Function value of the gridpoint with index $i$.\\
		$N$								&	Number of grid points.\\
		$u_{\ell, i}$					&	The surplus of the basis function of the point $x_{\ell, i}$.\\
		$\vec{\ell}$					&	A vector of levels. \\
		$\vec{h_\ell}$					&	A vector of mesh sizes. \\
		$\Omega_{\vec{\ell}}$			&	The grid with levels $\vec{\ell}$. \\
		$W_\ell$						&	Hierarchical increment space with level $\ell$. \\
		$I$								&	Index set. \\
		$V_\ell$						&	Nodal space of level $\ell$. \\
		$\tol$							&	The termination tolerance. \\
		$\refineTol$					&	The refinement tolerance. \\
		$v_\ell$						&	The volume of a basis function of level $\ell$. \\
		$\Omega^{(c)}_{\vec{\ell}}$		&	The combined \term{sparse grid} with level vector $\vec{\ell}$. \\
		$\lmin$							&	The minimum level in the \term{combination technique}. \\
		$\lmax$							&	The level of the \term{combination technique}. \\
		$U \bigoplus V$					&	The direct sum of U and V, i.e. ${ u + v : \forall u\in U,\, v\in V}$
		%\bottomrule
	\end{tabularx}
\end{center}

%Notation may differ from this in isolated cases, however this will be stated clearly where appropriate.